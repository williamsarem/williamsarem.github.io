% !TeX encoding = UTF-8
\documentclass[a4paper]{article}
\usepackage[T1]{fontenc}
\usepackage[left=1.9cm,right=1.9cm,top=1cm,bottom=2cm]{geometry}
\usepackage[utf8]{inputenc}
\usepackage[english]{babel}
\usepackage{hyperref}
\usepackage{array, multirow, makecell}
%\usepackage{xltabular}
\usepackage{xltabular}
\setlength{\parskip}{.3cm plus 2mm minus 1mm}

\title{CV}
\date{}

\newcolumntype{R}{>{\raggedright\arraybackslash}b{2.4cm}}
\begin{document}
	
	
	
	\noindent \LARGE William \textsc{Sarem}\\ \vspace{-.3cm}
	
	\noindent\normalsize Office 113 \hfill Year of birth: 1999\\ 
	Institut Fourier \hfill French nationality \\
	100, rue des maths \\
	38610 Gières
	
	\noindent Email address: \texttt{william.sarem@univ-grenoble-alpes.fr}   \\
	\noindent Website: \href{http://williamsarem.github.io}{http://williamsarem.github.io}
%	Phone number : +33~6~44~25~33~90\\ 
%	Statut : Normalien élève 
	
	\vspace{10pt}
	\noindent {\large \textbf{Education}}\linebreak\ 
	\vspace{-28pt}
	
	\noindent\rule{1.02\textwidth}{0.4pt}
		
	\begin{xltabular}{\textwidth}{ RX }
		2023~--~2026 & PhD Student at Institut Fourier, Grenoble Alpes University, supervised by Pierre Py.\\
		& \qquad Defended on 5 December 2025.\\
		& \qquad Title: Stein quotients via Patterson-Sullivan theory and the Shafarevich conjecture.\\
		2019~--~2023 & Student at École Normale Supérieure de Lyon. \\
		2021~--~2022 & Master's degree in Pure Mathematics (University of Strasbourg).\\
		%	2019 & Admission au Second concours Sciences exactes et experimentales de l'ENS de Lyon\\
		2019 & Bachelor's degree in Applied Mathematics (Paris Dauphine University).\\ 
		%	2018 &  CPES2 Majeure mathématiques~--~ Mineure physique  \\
		2017~--~2019 &  CPES Multidisciplinary Undergraduate degree: Bachelor's degree in Mathematics (Paris Sciences et Lettres University).
	\end{xltabular}
	

%	\vspace{10pt}
%	\noindent {\large \textbf{Recent Research internships}}\nobreak
%	\vspace{3pt}\hrule\vspace{6pt}
%	
%	
%	\begin{xltabular}{\textwidth}{ RX }
%		2023 & 6-month internship at Rennes University under the supervision of Christophe Dupont.\\ & \qquad Subject: Holomorphic properties of complex hyperbolic manifolds.\\
%		2022 & 4-month internship at ETH Zürich under the supervision of Marc Burger.\\ & \qquad Subject: Non-uniform lattices of $\mathrm{PU}(n,1)$. \\
%		2022 & 4-month internship at Strasbourg University under the supervision of Pierre Py.\\ & \qquad Subject: Stein manifolds and subgroups of $\mathrm{PU}(n,1)$.\\
%		2021 & 3-month internship at Bayreuth University under the supervision of Mihai Păun.\\ & \qquad Subject: Hodge theory and applications. \\
%		2020 & 6-week internship at Strasbourg University under the supervision of Pierre Py.\\ &\qquad  Subject : Milnor's fibration.
%	\end{xltabular}
%	
%	\vspace{10pt}
\vspace*{-.5cm}
	\noindent {\large \textbf{Preprints}}\linebreak\ 
\vspace{-28pt}

\noindent\rule{1.02\textwidth}{0.4pt}
	
	\begin{xltabular}{\textwidth}{ RX }
		2024 & \textit{Holomorphic functions on geometrically finite quotients of the ball}, preprint, submitted.\\
		2023 & \textit{Curvature properties and Shafarevich conjecture for toroidal compactifications of ball quotients}, to be published in Journal of Differential Geometry.
	\end{xltabular}
	

%	\vspace{50pt}
	\noindent {\large \textbf{{Research seminar talks}}}\\ 
\vspace{-28pt}

\noindent\rule{1.02\textwidth}{0.4pt}

	
	\begin{xltabular}{\textwidth}{ RX }
		November 2026 & \textit{Forthcoming talk} : Conference on Complex Geometry, \textbf{Daejeon, Korea}\\
		December 2025 & Séminaire Géométrie et Topologie, LMBA, \textbf{Brest}.\\
		&\qquad \textit{Entropie, convexité holomorphe et espaces localement symétriques.}\\
		November 2025 & Séminaire de Géométrie et Topologie, IMJ-PRG, \textbf{Paris}.\\
		&\qquad \textit{Sous-ensembles analytiques compacts dans des espaces quotients.}\\
		October 2025 & Séminaire de Géométrie, Groupes et Dynamique, UMPA, \textbf{Lyon}.\\
		&\qquad \textit{Sous-ensembles analytiques compacts dans des espaces quotients.}\\
		July 2025 & Oberseminar (2 x 90min) , Mathematisches Institut, \textbf{Bayreuth}. \\
		&\qquad \textit{Toroidal compactifications of ball quotients and Shafarevich conjecture.}\\
		April 2025 & Short talk: Geometric structures and discrete group actions, \textbf{CIRM}, Marseille.\\
		&\qquad \textit{Holomorphic properties of ball quotients: towards a higher rank generalization?}\\
		March 2025 & Séminaire Géométrie Complexe, IECL, \textbf{Nancy}.\\
		&\qquad \textit{Propriétés holomorphes des quotients de la boule et exposant critique.}\\
		January 2025 & Séminaire Algèbre, Topologie et Géométrie, LJAD, \textbf{Nice}.\\
		&\qquad \textit{Holomorphic properties of ball quotients.}\\
		January 2025 & Séminaire Géométrie Analytique, IRMAR, \textbf{Rennes}.\\
		&\qquad \textit{Propriétés holomorphes des quotients de la boule.}\\
		December 2024 & Séminaire Algèbre et Géométries, Institut Fourier, \textbf{Grenoble}.\\
		&\qquad \textit{Propriétés holomorphes des quotients de la boule.}\\				
		November 2024 & Séminaire Géométrie et Topologie, I2M, \textbf{Marseille}.\\
		&\qquad \textit{Propriétés holomorphes des quotients de la boule par un groupe d'exposant critique petit.}\\
		October 2023 & Séminaire Géométrie, IMB, \textbf{Bordeaux}. \\
		&\qquad \textit{Compactifications toroïdales de quotients de la boule et conjecture de Shafarevich.}\\
		October 2023 & Séminaire Algèbre et Géométries, Fourier Institute, \textbf{Grenoble}. \\
		&\qquad \textit{Conjecture de Shafarevich pour les compactifications toroïdales de quotients de la boule.} \\
		June 2023 &Séminaire Géométrie Analytique, IRMAR, \textbf{Rennes}.\\
		& \qquad \textit{Propriétés de courbure et conjecture de Shafarevich pour les compactifications} \\ &\qquad \textit{toroidales de quotients de la boule.} \\
		June 2023 & Short talk: Alpine meeting on nonpositive curvature in Kähler geometry, \textbf{Aussois}.\\
		& \qquad \textit{Curvature properties for toroidal compactifications of ball quotients.}\\
		June 2022 & GT3 seminar, IRMA, \textbf{Strasbourg}. \\
		& \qquad \textit{Variétés de Stein et sous-groupes de $\mathrm{PU}(n,1)$}.
	\end{xltabular}
	
	\noindent {\large \textbf{PhD seminar talks}}\\ 
	\vspace{-28pt}
	
	\noindent\rule{1.02\textwidth}{0.4pt}
	
	
	\begin{xltabular}{\textwidth}{ RX }
		June 2025 & UMPA, \textbf{ENS de Lyon}. \\
		&\qquad \textit{The Poincaré disk, its complex generalization and holomorphic functions.}\\
		June 2025 & Fourier Institute, \textbf{Grenoble}. \\
		&\qquad \textit{The Poincaré disk, its complex generalization and holomorphic functions.}\\
		March 2025 & IRMA, \textbf{Strasbourg}.\\
		&\qquad \textit{The Poincaré disk, its complex generalization and holomorphic functions.}.\\ 
		May 2024 & LMPA, \textbf{Calais}.\\
		&\qquad \textit{Le disque de Poincaré, sa généralisation complexe et leurs fonctions holomorphes.}\\
		April 2024 & Rencontres Doctorales Lebesgue, \textbf{Angers}.\\
		&\qquad \textit{The complex hyperbolic space and its holomorphic functions.}\\
		March 2024 & Fourier Institute, \textbf{Grenoble}. \\
		&\qquad \textit{The Poincaré disk, its complex generalization and holomorphic functions.}\\
		March 2023 & IRMAR, \textbf{Rennes}.\\
		& \qquad \textit{La formule de Matsushima.}\\
		December 2022 & \textbf{ETH Zürich}.\\
		& \qquad \textit{Harmonic forms and cohomology in smooth (complex) manifolds.}\\
	\end{xltabular}
	
%	\vspace{10pt}
	
	\noindent {\large \textbf{Research stays}}\\ 
	\vspace{-28pt}
	
	\noindent\rule{1.02\textwidth}{0.4pt}
	
	
	\begin{xltabular}{\textwidth}{ p{2.5cm}X }
		Jan.--Feb. 2026 & Visiting \textbf{Benoît Cadorel}, IECL, Nancy, 2 months.\\
		July 2025 & Visiting \textbf{Mihai Păun}, Mathematisches Institut, Bayreuth, 1 week.\\
		July 2025 &	Visiting \textbf{Christian Miebach}, ULCO, Calais, 1 week.\\
		May 2024 & Visiting \textbf{Christian Miebach}, ULCO, Calais, 3 days.\\
		January 2024 & Visiting \textbf{Simone Diverio}, la Sapienza, Rome, 1 week.
	\end{xltabular}
	
	
%	\vspace{10pt}
	
	\noindent {\large \textbf{Attended conferences}}\\ 
	\vspace{-28pt}
	
	\noindent\rule{1.02\textwidth}{0.4pt}

	\begin{xltabular}{\textwidth}{ RX }
		April 2026 & \textit{Forthcoming conference: } Complex Analysis, Oberwolfach, Germany.\\
		April 2025 & Geometric structures and discrete group actions, CIRM, Marseille.\\
		February 2025 & ANR meeting: L² Invariants, Grenoble.\\
		April 2024 & PhD meeting: Rencontres Doctorales Lebesgue, Angers.\\
		November 2023 & PhD meeting: Parole aux jeunes chercheuses et chercheurs du réseau Platon, Grenoble.\\
		July 2023 &Lattices in Negative Curvature, Les Diablerets.\\
		June 2023 & Alpine meeting on nonpositive curvature in Kähler geometry, Aussois.\\
		July 2022 & Complex Hyperbolic Geometry and Related Topics, CIRM, Marseille.
	\end{xltabular}
	
%	\vspace{10pt}	
	\noindent {\large \textbf{Teaching at Grenoble Alpes University}}\\ 
	\vspace{-28pt}
	
	\noindent\rule{1.02\textwidth}{0.4pt}

	\begin{xltabular}{\textwidth}{ RX }
		2025~--~2026 & $\bullet$ Lie Groups and Lattices (Master 2 course). 8h, Tutorials (in English).\\
		&$\bullet$ MAT101: Mathematical Language, Elementary Algebra and Geometry.\\ &\qquad 56h, Lectures and Tutorials.\\
		2024~--~2025 & $\bullet$ MAT236: Introduction to Mathematical Biology and Population Dynamics.\\ &\qquad 32h, Lectures and Tutorials.\\
					 & $\bullet$ MAT101: Mathematical Language, Elementary Algebra and Geometry.\\ &\qquad 32h, Lectures and Tutorials.\\
		2023~--~2024 & $\bullet$ MAT236: Introduction to Mathematical Biology and Population Dynamics.\\ &\qquad 32h, Lectures and Tutorials (in English).\\
					 & $\bullet$ MAT101: Mathematical Language, Elementary Algebra and Geometry.\\ &\qquad 32h, Lectures and Tutorials.\\
	\end{xltabular}


%\vspace{10pt}	
\noindent {\large \textbf{Responsabilities}}\\ 
\vspace{-28pt}

\noindent\rule{1.02\textwidth}{0.4pt}

\begin{xltabular}{\textwidth}{ RX }
	2025 &  $\bullet$ \textbf{PhD students' representative} at the doctoral school MSTII.\\
	2024~--~2025 &  $\bullet$ \textbf{Organizer of the Séminaire Compréhensible.}\\&\qquad This is the PhD seminar of the Fourier Institute.\\
& $\bullet$ \textbf{Organizer of the PhD days of the Fourier Institute.}\\ &\qquad This consists in 4 days of talks given by PhD students.\\
\end{xltabular}

\noindent {\large \textbf{Outreach}}\\ 
\vspace{-28pt}

\noindent\rule{1.02\textwidth}{0.4pt}

\begin{xltabular}{\textwidth}{ RX }
	2021 &  (with V. Issa) \textit{À la recherche des triplets pythagoriciens}, Image des Maths.\\
	2021 &  (with M. Passard) \textit{Pratique de la recherche en mathématiques}, J. de Math. de l'École, Lyon.\\
\end{xltabular}


	\vspace{10pt}
	\noindent {\large \textbf{Other scientific activities}}\\ 
	\vspace{-28pt}
	
	\noindent\rule{1.02\textwidth}{0.4pt}
	
	\begin{xltabular}{\textwidth}{ RX }
	2025 -- ... & Reviewer service, zbMATH Open.
	\end{xltabular}
	
	\vspace{10pt}
	\noindent {\large \textbf{Languages spoken}}\\ 
	\vspace{-28pt}
	
	\noindent\rule{1.02\textwidth}{0.4pt}
	
	\begin{xltabular}{\textwidth}{ RX }
		French & Native speaker \\	
		English & Fluent
	\end{xltabular}
	
\end{document}

